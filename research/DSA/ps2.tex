\documentclass[11pt,a4paper]{article}
%\usepackage{bbding}
%\fussy
\usepackage{framed, xcolor}
%\usepackage{tocvsec2}
%\usepackage{datetime}
%\usepackage{pifont}
%\usepackage{tikz}  
%\usepackage{pdflscape}
%\usepackage{subfigure}          % for subfigures in ACM
%\usepackage{colortbl}
%\usepackage{booktabs}
%\usepackage{pdfsync}
%\usepackage[pagebackref=true,citecolor=newblue,colorlinks=true,linkcolor=ocre,bookmarks=false]{hyperref}
%
%\usepackage[font=scriptsize,bf]{caption}
%\usepackage{tikz,subfigure}
%\usepackage[active]{srcltx}
%
%\usepackage[margin=1in]{geometry}
%\usepackage{algorithm}
%\usepackage{algorithmic}
%\usepackage{datetime}
\usepackage{epsfig,amssymb,amsfonts,amsmath,amsthm}
%\usepackage{multirow}
%\usepackage[numbers,sort&compress,sectionbib]{natbib}
%\bibliographystyle{alpha}
%\newcommand{\tsum}{\textstyle\sum}
%\newcommand{\tprod}{\textstyle\prod}
%\usepackage{amsfonts}
%
%\usepackage{xspace}
%\usepackage{tabularx}
%\usepackage{pstricks}
%\usepackage{setspace}
%%\ifx\bibfont\undefined\newcommand\bibfont\small\else\renewcommand\bibfont\small\fi
%\usepackage{xcolor}
%\newcommand{\minitab}[2][l]{\begin{tabular}{#1}#2\end{tabular}}
%\usepackage{rotating}
%\usepackage{minitoc}
%%\usepackage{newcent}\usepackage{tikz}
%\usepackage{enumerate}
%\definecolor{ocre}{rgb}{0.72,0,0} % Define the color BrickRed
%\definecolor{newblue}{rgb}{0.2,0.2,0.6} % Define the color BrickRed




%\usepackage[normalem]{ulem}
%\usepackage{fancybox}
%
% 
%\newenvironment{fminipage}%
%  {\begin{Sbox}\begin{minipage}}%
%  {\end{minipage}\end{Sbox}\fbox{\TheSbox}}
%  
%\newenvironment{algbox}[0]{\vskip 0.2in
%\noindent 
%\begin{fminipage}{6.3in}
%}{
%\end{fminipage}
%\vskip 0.2in
%}


\newcommand{\remove}[1]{}
%\newcommand{\ts}{}
\renewcommand{\textstyle}{}
\renewcommand{\d}{{\ensuremath{ \mathbf{d}}}}
\newcommand{\Geo}{\mathsf{Geo}}
\newcommand{\N}{\mathbb{N}}
\newcommand{\R}{\mathbb{R}}
\newcommand{\T}{\mathbb{T}}
\newcommand{\E}{\mathbb{E}}
\newcommand{\rot}{\intercal}
\newcommand{\ce}{\mathrm{e}}
\newcommand{\tr}{\mathrm{tr}}
\newcommand{\gen}{\mathcal{G}}
\newcommand{\calR}{\mathcal{R}}
\newcommand{\ext}{\mathsf{Ext}}
\newcommand{\cond}{\mathsf{Con}}
\newcommand{\calp}{\mathcal{P}}
\newcommand{\supp}{\mathrm{supp}}
\newcommand{\Oh}{O}
\newcommand{\n}{1}
\newcommand{\Ref}{\textcolor{red}{[ref]}}
\newcommand{\taucont}{\tau_{\cont}}
\newcommand{\poly}{\operatorname{poly}}
\newcommand{\polylog}{\operatorname{polylog}}
\renewcommand{\deg}{\mathrm{deg}}
\newcommand{\diam}{\operatorname{diam}}
\newcommand{\Odd}{\mathsf{Odd}}
\newcommand{\APT}{\mathsf{APT}}
\newcommand{\eps}{\epsilon}
\newcommand{\calX}{\mathcal{X}}
\newcommand{\hatc}{\hat{c}}
\newcommand{\hatx}{\hat{x}}
\newcommand{\core}{\mathsf{CORE}}

\newcommand{\vol}{\operatorname{vol}}
\newcommand{\sign}{\mathsf{sign}}
\newcommand{\de}{\operatorname{de}}
\newcommand{\even}{\mathsf{even}}
\newcommand{\odd}{\mathsf{odd}}
\newcommand{\COST}{\mathsf{COST}}
\newcommand{\OPT}{\mathsf{OPT}}
\newcommand{\ALG}{\textsf{ABC}}


\newcommand{\showproof}[1]{#1}
 


\newcommand{\argmin}{\operatorname{argmin}}
\newcommand{\argmax}{\operatorname{argmax}}
\newcommand{\mix}{\operatorname{mix}}
\DeclareMathOperator{\spn}{span}
\DeclareMathOperator{\dmn}{dim}
\renewcommand{\leq}{\leqslant}
\renewcommand{\geq}{\geqslant}
\renewcommand{\le}{\leqslant}
\renewcommand{\ge}{\geqslant}
\newcommand{\algref}[1]{Algorithm~\ref{alg:#1}}
\newcommand{\thmref}[1]{Theorem~\ref{thm:#1}}
\newcommand{\thmmref}[1]{Thm.~\ref{thm:#1}}
\newcommand{\thmrefs}[2]{Theorems~\ref{thm:#1} and~\ref{thm:#2}}
\newcommand{\proref}[1]{Proposition~\ref{pro:#1}}
\newcommand{\lemref}[1]{Lemma~\ref{lem:#1}}
\newcommand{\lemrefs}[2]{Lemmas~\ref{lem:#1} and~\ref{lem:#2}}
\newcommand{\lemrefss}[3]{Lemmas~\ref{lem:#1},~\ref{lem:#2}, and~\ref{lem:#3}}
\newcommand{\corref}[1]{Corollary~\ref{cor:#1}}
\newcommand{\obsref}[1]{Observation~\ref{obs:#1}}
\newcommand{\defref}[1]{Definition~\ref{def:#1}}
\newcommand{\defrefs}[2]{Definitions~\ref{def:#1} and~\ref{def:#2}}
\newcommand{\assref}[1]{Assumption~\eqref{ass:#1}}
\newcommand{\conref}[1]{Conjecture~\ref{con:#1}}
\newcommand{\figref}[1]{Figure~\ref{fig:#1}}
\newcommand{\figrefs}[2]{Figures~\ref{fig:#1} and~\ref{fig:#2}}
\newcommand{\tabref}[1]{Table~\ref{tab:#1}}
\newcommand{\secref}[1]{Section~\ref{sec:#1}}
\newcommand{\secrefs}[2]{Sections~\ref{sec:#1} and~\ref{sec:#2}}
\newcommand{\charef}[1]{Chapter~\ref{cha:#1}}
\newcommand{\eq}[1]{\eqref{eq:#1}}
\newcommand{\eqs}[2]{equations~\eqref{eq:#1} and~\eqref{eq:#2}}
\newcommand{\eqss}[3]{equations~\eqref{eq:#1},~\eqref{eq:#2}, and~\eqref{eq:#3}}
\newcommand{\Eqs}[2]{Equations~\eqref{eq:#1} and~\eqref{eq:#2}}
\newcommand{\propref}[1]{P\ref{prop:#1}}
\newcommand{\COOR}{\mathcal{PR}}
%\newcommand{\pro}[1]{\mathbf{Pr}[\,#1\,]}
\newcommand{\pr}[1]{\mathbf{Pr} [\,#1\,]}
\newcommand{\Pro}[1]{\mathbf{Pr} \left[\,#1\,\right]}
\newcommand{\Prob}[2]{\mathbf{Pr}_{#1} \left[\,#2\,\right]}
\newcommand{\PRO}[1]{\widetilde{\mathbf{Pr}} \left[\,#1\,\right]}
\newcommand{\Proo}[1]{\mathbf{Pr}[\,#1\,]}
\newcommand{\e}{\mathbf{E}}
\newcommand{\ex}[1]{\mathbf{E}[\,#1\,]}
\newcommand{\Ex}[1]{\mathbf{E} \left[\,#1\,\right]}
\newcommand{\EX}[1]{\widetilde{\mathbf{E}} \left[\,#1\,\right]}
\newcommand{\EXX}[2]{\mathbf{E}_{#1} \left[\,#2\,\right]}
\newcommand{\Var}[1]{\mathbf{Var} \left[\,#1\,\right]}
\newcommand{\Cov}[1]{\mathbf{Cov} \left[\,#1\,\right]}
%\newcommand{\EX}[1]{\mathbb{E}\Bigl[\,#1\,\Bigr]}
\newcommand{\degree}{\operatorname{deg}}
\newcommand{\girth}{\operatorname{girth}}
\renewcommand{\diam}{\operatorname{diam}}
\newcommand{\id}{\mathsf{id}}
\renewcommand{\tilde}{\widetilde}
\renewcommand{\epsilon}{\varepsilon}
\newcommand{\opt}{\mathrm{opt}}
\newcommand{\aseq}{\{A_i\}_{i=1}^k}
\newcommand{\sseq}{\{S_i\}_{i=1}^k}
\newcommand{\AlgoMean}{\mathsf{AlgoMean}}


\definecolor{ocre}{RGB}{150,22,11} % Define the orange color used for highlighting throughout the book
\newcommand{\hfamily}{\mathcal{H}}
\newcommand{\calL}{\mathcal{L}}
\newcommand{\calA}{\mathcal{A}}
\newcommand{\expn}{\mathrm{e}}




%\newcommand{\FOCS}[2]{#1 Annual IEEE Symposium on Foundations of Computer Science
%(FOCS'#2)}
%\newcommand{\STOC}[2]{#1 Annual ACM Symposium on Theory of Computing (STOC'#2)}
%\newcommand{\SODA}[2]{#1 Annual ACM-SIAM Symposium on Discrete Algorithms (SODA'#2)}
%\newcommand{\ICALP}[2]{#1 International Colloquium on Automata, Languages, and
%Programming (ICALP'#2)}
%\newcommand{\PODC}[2]{#1 Annual ACM-SIGOPT Principles of Distributed Computing
%(PODS'#2)}
%\newcommand{\STACS}[2]{#1 International Symposium on Theoretical Aspects of Computer
%Science (STACS'#2)}
%\newcommand{\SPAA}[2]{#1 Annual ACM Symposium on Parallel Algorithms and
%Architectures (SPAA'#2)}
%\newcommand{\MFCS}[2]{#1 International Symposium on Mathematical Foundations of
%Computer
%Science (MFCS'#2)}
%\newcommand{\ISAAC}[2]{#1 International Symposium on Algorithms and Computation
%(ISAAS'#2)}
%\newcommand{\WG}[2]{#1 Workshop of Graph-Theoretic Concepts in Computer Science
%(WG'#2)}
%\newcommand{\SIROCCO}[2]{#1 International Colloquium on Structural Information and
%Communication Complexity (SIROCCO'#2)}
%\newcommand{\IPDPS}[2]{#1 IEEE International Parallel and Distributed Processing
%Symposium (IPDPS'#2)}
%\newcommand{\RANDOM}[2]{#1 International Workshop on Randomization and Computation
%(RANDOM'#2)}
%\newcommand{\CCC}[2]{#1 Conference on Computational Complexity
%(CCC'#2)}
%\newcommand{\ICML}[2]{#1 International Conference on Machine Learning
%(ICML'#2)}
%

\usepackage{setspace}
\def\argmax{\operatornamewithlimits{argmax}}
\def\argmin{\operatornamewithlimits{argmin}}
\def\mod{\operatorname{mod}}

\newcommand{\phiin}{\phi_{\mathsf{in}}}
\newcommand{\phiout}{\phi_{\mathsf{out}}}


\newcommand{\Z}{{\mathbb{Z}}}
\newcommand{\setmid}{\,|\,}
\newcommand{\ee}{\varepsilon}


%\newcommand{\NOTE}[1]{\marginpar{\setstretch{0.43}\textcolor{red}{\bf\tiny#1}}}{\bf \tiny #1}}}



\title{Problem sheet 2}


\date{}

\begin{document}


\maketitle

\section*{Solutions (CLRS)}
Use the substitution method to verify your answer.


\noindent\textbf{4.4-1.} $T(n) = 3T(n / 2) + n$

Assume that
\begin{equation}
\label{eq-first}
T(k) \leq ck^{\log_2 3} -ck, \forall k < n
\end{equation}

We obtain:
\begin{equation*}
\begin{split}
T(n) & = 3c(n/2)^{\log_2 3} - c(n/2) + n \\
	 & \leq cn^{\log_2 3} - cn / 2 + n \\
\end{split}
\end{equation*}

Follows from $3/2^{\log_2 3} = 1$.

\begin{equation*}
	T(n) \leq cn^{\log_2 3}
\end{equation*}

The last inequality holds as long as:
\begin{align*}
	-cn/2 + n & \leq 0 \\
	c & \geq 2
\end{align*}
\break

In order for the induction (Eq.\ref{eq-first}) to hold pick $c = 5$ and $n \geq 2$; Base case is solved by $T(2) = 5 \leq 5 \cdot 2^{\log_2 3} - 5\cdot2$.


\noindent\textbf{4.4-2.} $T(n) = T(n / 2) + n^2$

Assume that
\begin{equation}
\label{eq-second}
T(k) \leq ck^2, \forall k < n
\end{equation}

\begin{equation}
\label{eq-v1}
\begin{split}
T(n) & \leq c(n/2)^{2} - (n/2)^2 \\
	 & \leq n^2(c+1)/4 \\
	 & \leq cn^2
\end{split}
\end{equation}

Eq.\ref{eq-v1} is true as long as $(c+1)/4 \leq c$, namely: $c \geq 1/3$. Choose $c=1$; Base case $T(1) = 1 \leq 1 \cdot 1$.


\noindent\textbf{4.3-3.} $T(n) = 2T(n / 2) + n \in O(n \log{n})$. Prove that $T(n) \in \Omega(n \log{n})$.

Assume that
\begin{equation}
\label{eq-third}
T(k) \geq c \cdot k \log{k}
\end{equation}

\begin{equation}
\label{eq-v2}
\begin{split}
	T(n)&= 2 (cn/2) \log{n/2} + n \\
		&= cn\log{n} - cn\log{2} + n \\
		& \geq cn\log{n}
\end{split}
\end{equation}

Eq. \ref{eq-v2} is true as long as $-cn \log 2 + n \geq 0$, namely: $c \leq 1$. Pick $c = 1$ and see that base case holds.


\noindent\textbf{4.5-1.} Use Master method to solve the following:

\begin{description} \itemsep8pt
	\item $T(n) = 2T(n/4) + 1$. $a = 2$, $b=4$, $f(n) = 1 \in O(n^{\log_4 2 - 0.5})$. Apply case $1$ to get $T(n) \in \Theta(n^{\log_4{2}}) = \Theta(n^{1/2})$

	\item $T(n) = 2T(n/4) + \sqrt{n}$. $a=2$, $b=4$, $f(n) \in \Theta(n^{1/2})$. Apply case $2$ to get $T(n) \in \Theta(\sqrt{n}\log{n})$

	\item $T(n) = 2T(n/4) + n$. $a = 2$, $b=4$, $f(n) = \Theta(n^{\log_4 2 + 0.5})$. Apply case $3$ after checking that there is a positive $c < 1$ for which $2f(n/4) \leq cf(n)$. This implies that $T(n) \in \Theta(n)$.

	\item $T(n) = 2T(n/4) + n^2$. $a=2$, $b=4$, $f(n) \in \Theta(n^{\log_4 2 + 1.5})$. Apply case $3$ and see that $c=1/9 < 1$. This implies that $T(n) \in \Theta(n^2)$.

\end{description}


\section*{Solutions (POA)}
Solve the following recurrences exactly:

\noindent\textbf{205.} $T(1) = 3$. $T(n) = T(n - 1) + 2n - 3$.

Sum all the levels in the tree. We claim that:
\begin{equation}
	T(k) = \sum_{i=2}^{k}(2i - 3) + 3 = k^2 - 2k + 4, \forall 1 \leq k < n
\end{equation}

Base case holds since $T(1) = 1^2 - 2 \cdot 1 + 4 = 3$.

Now $T(n) = (n-1)^2 + 2n - 3 = n^2 - 2n + 4 \in \Theta(n^2)$.

\noindent\textbf{206.} $T(1) = 1$. $T(n) = 2T(n - 1) + n - 1$.

Notice that if we sum all the levels in the tree we get:
\begin{equation}
	T(k)= k-1 + 2(k-2) + 2^2(k-3) + \dots + 2^{k-2} + 2^{k-1}
\end{equation}

In these situations we guess that $T(n) \in \Theta(2^n)$ and then prove with the substitution method: $T(n) \in O(2^n)$ and $T(n) \in \Omega(2^n)$.

Assume that:
\begin{equation}
\label{eq-expo}
T(k) \leq c2^k - ck, \forall k < n
\end{equation}

We obtain that:
\begin{equation}
\label{eq-expo2}
\begin{split}
T(n) & \leq 2c2^{n-1}+n-1 - c(n-1) \\
	&= c2^{n} + (n-1)(1 - c) \\
	& \leq c2^{n}
\end{split}
\end{equation}

Eq. $\ref{eq-expo2}$ follows if $(n-1)(1-c) \leq 0$. For this choose $c=2$. Base case holds since $T(1) = 1 \leq 2 \cdot 2^{1} - 2\cdot 1=2$.

Assume that:
\begin{equation}
\label{eq-expo3}
T(k) \geq c2^k, \forall k < n
\end{equation}

We obtain that:
\begin{equation}
\begin{split}
	T(n) \geq 2c2^{n-1}+n-1 \\
	 & \geq c2^{n} + n-1 \\
	 & \geq c2^{n}
\end{split}
\end{equation}

This is true since $n \geq 1$. Now choose $c = 1$. Base case holds for $T(1) = 1 \geq 1 \cdot 2^{1} = 1$.

\noindent\textbf{212.} $T(1) = 3$. $T(n) = 4T(n/3) + 2n - 1$. $n = 3^p$, $p \geq 0$.

Using the tree method and summing on all levels we get that:
\begin{equation*}
	T(n) = \sum_{i = 0}^{\log_3 n - 1}((4/3)^i \cdot 2 n - 1) + 4^{\log_3 n}
\end{equation*}


Observe that the last level has $n^{\log_3 4}$ vertices. This will turn out a valid guess and we can prove using the substitution method that $T(n) \in O(n^{\log_3 4})$ as well as $T(n) \in \Omega(n^{\log_3 4})$ concluding with $T(n) \in \Theta(n^{\log_3 4})$.


\noindent\textbf{218.} $T(1) = 1$. $T(n) = 4T(n/3) + n^2$. $n = 3^p$, $p \geq 0$.

\begin{equation}
\begin{split}
	\sum_{i = 0}^{\log_3 n}{(\frac{n}{3^i}^2 \cdot 4^i)} &= n^2 \sum_{i = 0}^{\log_3 n}{\frac{4^i}{3^{2i}}} \\
	&= n^2(1 - (4/9)^{\log_3 n}) / (1 - 4/9) \\
	&= n^2 (1 - (4/9)^{\log_3 n}) \cdot 5/9 \in \Theta(n^2)
\end{split}
\end{equation}

\noindent\textbf{220.} $T(1) = 1$. $T(n) = T(n/4) + \sqrt{n} + 1$. $n = 4^k$, $k \geq 0$.

\begin{equation}
\begin{split}
	\sum_{i=0}^{\log_4 n}(\sqrt n/2^i + 1) + \log_4 n= \sqrt n (2^n - 1) / 2^{n+1} + 2 \log_4 n
\end{split}
\end{equation}
As $(2^n - 1) / 2^{n+1} \in (0, 1)$, $T(n) \in \Theta(\sqrt n)$
\end{document}