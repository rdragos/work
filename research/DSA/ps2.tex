\documentclass{article}
\usepackage[utf8]{inputenc}
\usepackage{parskip}
\usepackage{enumerate}
\usepackage{fancybox}
\usepackage{tikz}
\usepackage[left=3cm, right=3cm]{geometry}

%\usepackage{bbding}
%\fussy
\usepackage{framed, xcolor}
%\usepackage{tocvsec2}
%\usepackage{datetime}
%\usepackage{pifont}
%\usepackage{tikz}  
%\usepackage{pdflscape}
%\usepackage{subfigure}          % for subfigures in ACM
%\usepackage{colortbl}
%\usepackage{booktabs}
%\usepackage{pdfsync}
%\usepackage[pagebackref=true,citecolor=newblue,colorlinks=true,linkcolor=ocre,bookmarks=false]{hyperref}
%
%\usepackage[font=scriptsize,bf]{caption}
%\usepackage{tikz,subfigure}
%\usepackage[active]{srcltx}
%
%\usepackage[margin=1in]{geometry}
%\usepackage{algorithm}
%\usepackage{algorithmic}
%\usepackage{datetime}
\usepackage{epsfig,amssymb,amsfonts,amsmath,amsthm}
%\usepackage{multirow}
%\usepackage[numbers,sort&compress,sectionbib]{natbib}
%\bibliographystyle{alpha}
%\newcommand{\tsum}{\textstyle\sum}
%\newcommand{\tprod}{\textstyle\prod}
%\usepackage{amsfonts}
%
%\usepackage{xspace}
%\usepackage{tabularx}
%\usepackage{pstricks}
%\usepackage{setspace}
%%\ifx\bibfont\undefined\newcommand\bibfont\small\else\renewcommand\bibfont\small\fi
%\usepackage{xcolor}
%\newcommand{\minitab}[2][l]{\begin{tabular}{#1}#2\end{tabular}}
%\usepackage{rotating}
%\usepackage{minitoc}
%%\usepackage{newcent}\usepackage{tikz}
%\usepackage{enumerate}
%\definecolor{ocre}{rgb}{0.72,0,0} % Define the color BrickRed
%\definecolor{newblue}{rgb}{0.2,0.2,0.6} % Define the color BrickRed




%\usepackage[normalem]{ulem}
%\usepackage{fancybox}
%
% 
%\newenvironment{fminipage}%
%  {\begin{Sbox}\begin{minipage}}%
%  {\end{minipage}\end{Sbox}\fbox{\TheSbox}}
%  
%\newenvironment{algbox}[0]{\vskip 0.2in
%\noindent 
%\begin{fminipage}{6.3in}
%}{
%\end{fminipage}
%\vskip 0.2in
%}


\newcommand{\remove}[1]{}
%\newcommand{\ts}{}
\renewcommand{\textstyle}{}
\renewcommand{\d}{{\ensuremath{ \mathbf{d}}}}
\newcommand{\Geo}{\mathsf{Geo}}
\newcommand{\N}{\mathbb{N}}
\newcommand{\R}{\mathbb{R}}
\newcommand{\T}{\mathbb{T}}
\newcommand{\E}{\mathbb{E}}
\newcommand{\rot}{\intercal}
\newcommand{\ce}{\mathrm{e}}
\newcommand{\tr}{\mathrm{tr}}
\newcommand{\gen}{\mathcal{G}}
\newcommand{\calR}{\mathcal{R}}
\newcommand{\ext}{\mathsf{Ext}}
\newcommand{\cond}{\mathsf{Con}}
\newcommand{\calp}{\mathcal{P}}
\newcommand{\supp}{\mathrm{supp}}
\newcommand{\Oh}{O}
\newcommand{\n}{1}
\newcommand{\Ref}{\textcolor{red}{[ref]}}
\newcommand{\taucont}{\tau_{\cont}}
\newcommand{\poly}{\operatorname{poly}}
\newcommand{\polylog}{\operatorname{polylog}}
\renewcommand{\deg}{\mathrm{deg}}
\newcommand{\diam}{\operatorname{diam}}
\newcommand{\Odd}{\mathsf{Odd}}
\newcommand{\APT}{\mathsf{APT}}
\newcommand{\eps}{\epsilon}
\newcommand{\calX}{\mathcal{X}}
\newcommand{\hatc}{\hat{c}}
\newcommand{\hatx}{\hat{x}}
\newcommand{\core}{\mathsf{CORE}}

\newcommand{\vol}{\operatorname{vol}}
\newcommand{\sign}{\mathsf{sign}}
\newcommand{\de}{\operatorname{de}}
\newcommand{\even}{\mathsf{even}}
\newcommand{\odd}{\mathsf{odd}}
\newcommand{\COST}{\mathsf{COST}}
\newcommand{\OPT}{\mathsf{OPT}}
\newcommand{\ALG}{\textsf{ABC}}


\newcommand{\showproof}[1]{#1}
 


\newcommand{\argmin}{\operatorname{argmin}}
\newcommand{\argmax}{\operatorname{argmax}}
\newcommand{\mix}{\operatorname{mix}}
\DeclareMathOperator{\spn}{span}
\DeclareMathOperator{\dmn}{dim}
\renewcommand{\leq}{\leqslant}
\renewcommand{\geq}{\geqslant}
\renewcommand{\le}{\leqslant}
\renewcommand{\ge}{\geqslant}
\newcommand{\algref}[1]{Algorithm~\ref{alg:#1}}
\newcommand{\thmref}[1]{Theorem~\ref{thm:#1}}
\newcommand{\thmmref}[1]{Thm.~\ref{thm:#1}}
\newcommand{\thmrefs}[2]{Theorems~\ref{thm:#1} and~\ref{thm:#2}}
\newcommand{\proref}[1]{Proposition~\ref{pro:#1}}
\newcommand{\lemref}[1]{Lemma~\ref{lem:#1}}
\newcommand{\lemrefs}[2]{Lemmas~\ref{lem:#1} and~\ref{lem:#2}}
\newcommand{\lemrefss}[3]{Lemmas~\ref{lem:#1},~\ref{lem:#2}, and~\ref{lem:#3}}
\newcommand{\corref}[1]{Corollary~\ref{cor:#1}}
\newcommand{\obsref}[1]{Observation~\ref{obs:#1}}
\newcommand{\defref}[1]{Definition~\ref{def:#1}}
\newcommand{\defrefs}[2]{Definitions~\ref{def:#1} and~\ref{def:#2}}
\newcommand{\assref}[1]{Assumption~\eqref{ass:#1}}
\newcommand{\conref}[1]{Conjecture~\ref{con:#1}}
\newcommand{\figref}[1]{Figure~\ref{fig:#1}}
\newcommand{\figrefs}[2]{Figures~\ref{fig:#1} and~\ref{fig:#2}}
\newcommand{\tabref}[1]{Table~\ref{tab:#1}}
\newcommand{\secref}[1]{Section~\ref{sec:#1}}
\newcommand{\secrefs}[2]{Sections~\ref{sec:#1} and~\ref{sec:#2}}
\newcommand{\charef}[1]{Chapter~\ref{cha:#1}}
\newcommand{\eq}[1]{\eqref{eq:#1}}
\newcommand{\eqs}[2]{equations~\eqref{eq:#1} and~\eqref{eq:#2}}
\newcommand{\eqss}[3]{equations~\eqref{eq:#1},~\eqref{eq:#2}, and~\eqref{eq:#3}}
\newcommand{\Eqs}[2]{Equations~\eqref{eq:#1} and~\eqref{eq:#2}}
\newcommand{\propref}[1]{P\ref{prop:#1}}
\newcommand{\COOR}{\mathcal{PR}}
%\newcommand{\pro}[1]{\mathbf{Pr}[\,#1\,]}
\newcommand{\pr}[1]{\mathbf{Pr} [\,#1\,]}
\newcommand{\Pro}[1]{\mathbf{Pr} \left[\,#1\,\right]}
\newcommand{\Prob}[2]{\mathbf{Pr}_{#1} \left[\,#2\,\right]}
\newcommand{\PRO}[1]{\widetilde{\mathbf{Pr}} \left[\,#1\,\right]}
\newcommand{\Proo}[1]{\mathbf{Pr}[\,#1\,]}
\newcommand{\e}{\mathbf{E}}
\newcommand{\ex}[1]{\mathbf{E}[\,#1\,]}
\newcommand{\Ex}[1]{\mathbf{E} \left[\,#1\,\right]}
\newcommand{\EX}[1]{\widetilde{\mathbf{E}} \left[\,#1\,\right]}
\newcommand{\EXX}[2]{\mathbf{E}_{#1} \left[\,#2\,\right]}
\newcommand{\Var}[1]{\mathbf{Var} \left[\,#1\,\right]}
\newcommand{\Cov}[1]{\mathbf{Cov} \left[\,#1\,\right]}
%\newcommand{\EX}[1]{\mathbb{E}\Bigl[\,#1\,\Bigr]}
\newcommand{\degree}{\operatorname{deg}}
\newcommand{\girth}{\operatorname{girth}}
\renewcommand{\diam}{\operatorname{diam}}
\newcommand{\id}{\mathsf{id}}
\renewcommand{\tilde}{\widetilde}
\renewcommand{\epsilon}{\varepsilon}
\newcommand{\opt}{\mathrm{opt}}
\newcommand{\aseq}{\{A_i\}_{i=1}^k}
\newcommand{\sseq}{\{S_i\}_{i=1}^k}
\newcommand{\AlgoMean}{\mathsf{AlgoMean}}


\definecolor{ocre}{RGB}{150,22,11} % Define the orange color used for highlighting throughout the book
\newcommand{\hfamily}{\mathcal{H}}
\newcommand{\calL}{\mathcal{L}}
\newcommand{\calA}{\mathcal{A}}
\newcommand{\expn}{\mathrm{e}}




%\newcommand{\FOCS}[2]{#1 Annual IEEE Symposium on Foundations of Computer Science
%(FOCS'#2)}
%\newcommand{\STOC}[2]{#1 Annual ACM Symposium on Theory of Computing (STOC'#2)}
%\newcommand{\SODA}[2]{#1 Annual ACM-SIAM Symposium on Discrete Algorithms (SODA'#2)}
%\newcommand{\ICALP}[2]{#1 International Colloquium on Automata, Languages, and
%Programming (ICALP'#2)}
%\newcommand{\PODC}[2]{#1 Annual ACM-SIGOPT Principles of Distributed Computing
%(PODS'#2)}
%\newcommand{\STACS}[2]{#1 International Symposium on Theoretical Aspects of Computer
%Science (STACS'#2)}
%\newcommand{\SPAA}[2]{#1 Annual ACM Symposium on Parallel Algorithms and
%Architectures (SPAA'#2)}
%\newcommand{\MFCS}[2]{#1 International Symposium on Mathematical Foundations of
%Computer
%Science (MFCS'#2)}
%\newcommand{\ISAAC}[2]{#1 International Symposium on Algorithms and Computation
%(ISAAS'#2)}
%\newcommand{\WG}[2]{#1 Workshop of Graph-Theoretic Concepts in Computer Science
%(WG'#2)}
%\newcommand{\SIROCCO}[2]{#1 International Colloquium on Structural Information and
%Communication Complexity (SIROCCO'#2)}
%\newcommand{\IPDPS}[2]{#1 IEEE International Parallel and Distributed Processing
%Symposium (IPDPS'#2)}
%\newcommand{\RANDOM}[2]{#1 International Workshop on Randomization and Computation
%(RANDOM'#2)}
%\newcommand{\CCC}[2]{#1 Conference on Computational Complexity
%(CCC'#2)}
%\newcommand{\ICML}[2]{#1 International Conference on Machine Learning
%(ICML'#2)}
%

\usepackage{setspace}
\def\argmax{\operatornamewithlimits{argmax}}
\def\argmin{\operatornamewithlimits{argmin}}
\def\mod{\operatorname{mod}}

\newcommand{\phiin}{\phi_{\mathsf{in}}}
\newcommand{\phiout}{\phi_{\mathsf{out}}}


\newcommand{\Z}{{\mathbb{Z}}}
\newcommand{\setmid}{\,|\,}
\newcommand{\ee}{\varepsilon}


%\newcommand{\NOTE}[1]{\marginpar{\setstretch{0.43}\textcolor{red}{\bf\tiny#1}}}{\bf \tiny #1}}}



\title{Problem sheet 2}


\date{}

\begin{document}


\maketitle

\section*{Solutions (CLRS)}

\noindent\textbf{4.4-1.} $T(n) = 3T(n / 2) + n$.

Solving this type of problems 
involves two steps. Step 1, figure out what is a good solution for the recurrence. Here we can use telescoping sums, recurrence trees...or experience. In Step 2, we need to prove that our guess for the bound is correct. Alternatively, if we master the Master Theorem we can use it (whenever applicable) to determine a precise bound to the recurrence relation and don't need to provide a proof. Below we use the Master theorem to determine the bound and do the proof by the substitution method.

Using the Master theorem we have that $T(n) = \Theta(n\log_2{3})$ - indeed, $n \in \Oh(n^{\log_2{3} - \epsilon})$ for some $\epsilon > 0$ which is true since $\log_2{3} > 1$, so we are in the first case of the master theorem.

Next we would like to show by strong induction that this is indeed true. If we attempt a direct proof by strong induction that.

\begin{align*} T(n) \leq cn^{\log_{2}3}
\end{align*}
(for all large enough n) we will see that it does not quite work. See also the example in CLRS which explains why sometime it helps to make a stronger induction hypothesis then the one we actually try to prove.

The reason is that the induction hypothesis $T(k) \leq ck^{\log_2{3}}$ for all $k < n$ will not be sufficient to prove the desired relation for $n$, so instead of proving that for all large enough $n$,

\begin{equation*}
T(n) \leq c(n^{\log_2{3}} - n)
\end{equation*}

We first carry out the inductive step. So, assume that for any $k < n$ it holds that

\begin{align*} T(k) \leq c(k^{\log_2 3} -k)
\end{align*}

and using this assumption we show that $T(n) \leq cn^{\log_2 3} - n$. Notice that the above assumption implies that (by setting $k = n/2$) we assume that $T(n/2) \leq c(n/2)^{\log_2{3}} - cn/2$.

Our goal is to upper bound $T(n)$. We have that
\begin{equation*}
\begin{split}
T(n) &= 3T(n/2)+n \\
	 &= 3c(n/2)^{\log_2 3} - 3c(n/2) + n \\
	 & \leq cn^{\log_2 3} - 3cn / 2 + n \\
\end{split}
\end{equation*}

Above we have used the induction hypothesis and that  $3/2^{\log_2 3} = 1$.

To obtain what we want, that is:
\begin{equation*}
	T(n) \leq cn^{\log_2 3} - n
\end{equation*}

it is enough to set $c$ to any positive constant which satisfies the last inequality as long as:
\begin{align*}
	- 3cn / 2 + n \leq -cn
\end{align*}

so we can set $c$ to any value greater than $2$.

To complete the proof we also need to show that the base case for induction holds. As usual, we assume that if $T(1)$ is not specified then $T(1) = 1$, so it is easy to see that we cannot set the base for the induction to $n_0 = 1$. Instead, we set $n_0 = 2$. If we set $c = 5$ and $n \geq 2$ we get the desired inequality: $T(2) = 5 \leq 5 \cdot 2^{\log_2 3} - 5\cdot2$.


\noindent\textbf{4.4-2.} $T(n) = T(n / 2) + n^2$. We proceed by stron ginduction to show that $T(n) \in \Oh(n^2)$, that is $T(n) \leq cn^2$ for a well chosen c and for all sufficiently large $n$.

For the inductive step we assume that:

\begin{equation*}
\label{eq-second}
T(k) \leq ck^2, \forall k < n
\end{equation*}

which in particular means that we assume that $T(n/2) \leq c(n/2)^2$. We now upperbound $T(n)$. We have that:

\begin{equation*}
\label{eq-v1}
\begin{split}
T(n) &= T(n/2) + n^2 \\
	 & \leq c(n/2)^{2} + n^2 \\
	 & \leq n^2(c+1)/4 \\
\end{split}
\end{equation*}

We therefore get that $T(n) \leq cn^2$ for any $c$ for which $(c+1)/4 \leq c$, namely: $c \geq 4/3$. 

The base case holds for $n_0=1$ since $T(1) = 1 \leq 1 \cdot 1$. If we set $c$ to, say, $1$ we get that $T(n) \leq n^2$ for any $n$.


\noindent\textbf{4.3-3.} $T(n) = 2T(n / 2) + n \in O(n \log{n})$. Prove that $T(n) \in \Omega(n \log{n})$.

Using recurrences trees or the Master theorem we get that $T(n) = \Theta(n \log n)$. So we would need to prove both that $T(n) = \Oh(n \log n)$ and that $T(n) \in \Omega(n \log n)$. Here we prove the latter since the former property can be shown analogously.

We therefore show that $T(n) \geq c \cdot n \log n$ for some $c$ and for all large enough $n$. As usual we proceed by strong induction; we deal with the inductive step first. We assume that

Assume that
\begin{equation*}
\label{eq-third}
T(k) \geq c \cdot k \log{k}
\end{equation*}

and try to lowerbound $T(n)$. This is easy since:

\begin{equation*}
\label{eq-v2}
\begin{split}
	T(n) &= 2T(n/2) + n \\
		& \geq 2 (cn/2) \log{n/2} + n \\
		&= cn\log{n} - cn\log{2} + n \\
\end{split}
\end{equation*}

We get that $T(n) \geq cn \log n$ (as desired) for any $c$ such that $-cn \log 2 + n \geq 0$.

The base case holds for $n_0 = 1$: $T(1) = 1 \geq c \cdot 1 \log{1}$ for any $c$. We can therefore set $c = 1$ and $n_0 = 1$.

\noindent\textbf{4.5-1.} Use Master method to solve the following:

\begin{description} \itemsep8pt
	\item $T(n) = 2T(n/4) + 1$. $a = 2$, $b=4$, $f(n) = 1 \in O(n^{\log_4 2 - 0.5})$. Apply case $1$ to get $T(n) \in \Theta(n^{\log_4{2}}) = \Theta(n^{1/2})$

	\item $T(n) = 2T(n/4) + \sqrt{n}$. $a=2$, $b=4$, $f(n) \in \Theta(n^{1/2})$. Apply case $2$ to get $T(n) \in \Theta(\sqrt{n}\log{n})$

	\item $T(n) = 2T(n/4) + n$. $a = 2$, $b=4$, $f(n) = \Theta(n^{\log_4 2 + 0.5})$. Apply case $3$ after checking that there is a positive $c < 1$ for which $2f(n/4) \leq cf(n)$. This implies that $T(n) \in \Theta(n)$.

	\item $T(n) = 2T(n/4) + n^2$. $a=2$, $b=4$, $f(n) \in \Theta(n^{\log_4 2 + 1.5})$. Apply case $3$ and see that the inequality $2 \cdot f(n/4) \leq c \cdot f(n)$ holds iff. $c \geq 1/8$. This implies that $T(n) \in \Theta(n^2)$.

\end{description}


\section*{Solutions (POA)}
Solve the following recurrences exactly. Here the book requires that we should find closed form formulas for the function $T$ defined through the given recurrences. Howevere, it is sufficient for the purposes of this class that we find tight asymptotic approximation for the functions. These solutions show how to find the closed forms for the simpler recurrences and show how to get the bounds for the rest.

However, we will give the exact form for all the exercises as well as two examples solved by the recurrence tree method.

\noindent\textbf{205.} $T(1) = 3$. $T(n) = T(n - 1) + 2n - 3$.

Drawing the recursion tree when $k < n$ where $k \geq 2 $, specifying the amount of work done on each recursive step.\\\\
\begin{tikzpicture}[level/.style={sibling distance=60mm/#1}]
\node [circle,draw] (l1){$k$}
  child {node [circle,draw] (l2) {$k-1$}
  	child {node [circle,draw] (l3) {$k-2$}
  		child {node (l4) {$\vdots$}
  			child {node [circle,draw] (l5) {$1$}
  				child [grow=right] {node (eq5) {$=$} edge from parent[draw=none]
  					child [grow=right] {node (f5) {$3$} edge from parent[draw=none]
  						child [grow=up] {node (f4) {$\vdots$} edge from parent[draw=none]
  							child [grow=up] {node (f3) {$2(k-2)-3$} edge from parent[draw=none]
  								child [grow=up] {node (f2) {$2(k-1)-3$} edge from parent[draw=none]
  									child [grow=up] {node (f1) {$2k-3$} edge from parent[draw=none]}}}}}}}}}};
\path (l1) -- (f1) node [midway] {=};
\path (l2) -- (f2) node [midway] {=};
\path (l3) -- (f3) node [midway] {=};
\end{tikzpicture}\\
So the total work is the sum of the work done over all nodes. This is the the sum of the base cases, which here is just one that costs $3$ and the sum from layer 2 to the height of the tree of the work done on each layer. Since each layer reduces the problem by 1 the height of the tree is $k$. Also, since the problem is only broken down in to one sub problem, we only need one copy of the work for each layer. Therefore the overall work is as follows:
$$
T(k) = \left( \sum_{i=2}^{k}{2i-3} \right) + 3 =
\left( 2\sum_{i=2}^{k}{i} - \sum_{i=2}^{k}{3} \right) + 3 = 
$$
$$
2\frac{k(k+1)}{2} - k \cdot 3 + 4 =
k(k+1) - k \cdot 3 + 4 = k^2 + k - 3k + 4 = k^2 -2k + 4$$

To sum up, we get that:
\begin{equation*}
	T(k) = k^2 - 2k + 4, \forall 1 \leq k < n
\end{equation*}

For the induction to hold, we need to prove that the base case stands:
This is indeed the case, since $T(1) = 1^2 - 2 \cdot 1 + 4 = 3$.

Next, we know that $T(n) = T(n - 1) + 2n - 3$. Following the inductive step with $k=n-1$, we obtain that $T(n)=(n-1)^2 + 2n - 3 = n^2 - 2n + 4$. Thus $T(n) \in \Theta(n^2)$ because $n^2$ is the dominant term.

\noindent\textbf{206.} $T(1) = 1$. $T(n) = 2T(n - 1) + n - 1$.

To get a better intuition on what is happening in the recursion we draw the recurrence tree. Remember that each node labeled with a number $k$ denotes the function $T$ called with input $k$. Starting with a call to $T(k)$:

\begin{tikzpicture}[level/.style={sibling distance=60mm/#1}]
\node [circle,draw] (z){$k < n$}
  child {node [circle,draw] (a) {$k-1$}
    child {node [circle,draw] (b) {$k-2$}
      child {node {$\vdots$}
        child {node [circle,draw] (d) {$1$}}
        child {node [circle,draw] (e) {$1$}}
      } 
      child {node {$\vdots$}}
    }
    child {node [circle,draw] (g) {$k-2$}
      child {node {$\vdots$}}
      child {node {$\vdots$}}
    }
  }
  child {node [circle,draw] (j) {$k-1$}
    child {node [circle,draw] (k) {$k-2$}
      child {node {$\vdots$}}
      child {node {$\vdots$}}
    }
  child {node [circle,draw] (l) {$k-2$}
    child {node {$\vdots$}}
    child {node (c){$\vdots$}
      child {node [circle,draw] (o) {$1$}}
      child {node [circle,draw] (p) {$1$}
        child [grow=right] {node (q) {$=$} edge from parent[draw=none]
          child [grow=right] {node (q) {$1 \cdot 2^{k-1}$} edge from parent[draw=none]
            child [grow=up] {node (r) {$\vdots$} edge from parent[draw=none]
              child [grow=up] {node (s) {$4(k-3)=2^2(k-3)$} edge from parent[draw=none]
                child [grow=up] {node (t) {$2(k-2)=2^1(k-2)$} edge from parent[draw=none]
                  child [grow=up] {node (u) {$k-1=2^0(k-1)$} edge from parent[draw=none]}
                }
              }
            }
          }
        }
      }
    }
  }
};
\path (a) -- (j) node [midway] {+};
\path (b) -- (g) node [midway] {+};
\path (k) -- (l) node [midway] {+};
\path (k) -- (g) node [midway] {+};
\path (d) -- (e) node [midway] {+};
\path (o) -- (p) node [midway] {+};
\path (o) -- (e) node (x) [midway] {$\cdots$};
\path (q) -- (r) node [midway] {+};
\path (s) -- (r) node [midway] {+};
\path (s) -- (t) node [midway] {+};
\path (s) -- (l) node [midway] {=};
\path (t) -- (u) node [midway] {+};
\path (z) -- (u) node [midway] {=};
\path (j) -- (t) node [midway] {=};
\path (e) -- (x) node [midway] {+};
\path (o) -- (x) node [midway] {+};
\path (r) -- (c) node [midway] {$\cdots$};
\end{tikzpicture}\\

Notice that if we sum all the levels in the tree we get:
\begin{equation}
	T(k)= k-1 + 2(k-2) + 2^2(k-3) + \dots + 2^{k-2} + 2^{k-1}
\end{equation}

This can be viewed as:
\begin{equation*}
\label{eq-expo2}
\begin{split}
T(k)&= \sum_{i=0}^{k-2}(k - i - 1) \cdot 2^{i} + 2^{k-1} = \sum_{i=0}^{k-2}(k \cdot 2^{i} - i \cdot 2^{i} - 2^{i}) + 2^{k-1} \\
	&= k \cdot \sum_{i=0}^{k-2} 2^{i} - \sum_{i=0}^{k-2} i \cdot 2^{i} - \sum_{i=0}^{k-2}2^i + 2^{k-1} \\
	&=  k \cdot (2^{k-1} - 1) - (2^{k-1}\cdot k - 3\cdot 2^{k-1}+2) - (2^{k-1} - 1) + 2^{k-1} \\
	&= k \cdot 2^{k-1} - k - 2^{k-1}\cdot k + 3\cdot 2^{k-1}-2 -2^{k-1} + 1 + 2^{k-1}\\
	&= 3 \cdot 2^{k-1} - k - 1
\end{split}
\end{equation*}

One has to check the base case now. Note that the formula holds for every $k \geq 2$. On one hand, applying the formula gives us:
$T(2) = 3 \cdot 2^{1} - 2 - 1 = 3$.
On the other hand, the original recurrence gives us $T(2) = 2 \cdot 1 + 2 - 1 = 3$. In conclusion, the base case holds hence $T(n) \in \Theta(2^n)$ since $2^{n}$


In some situations we might not be brave enough to compute exactly $T(k)$ as before. So what remains to be done is to 'guess' that $T(n) \in \Theta(2^n)$ and then prove with the substitution method: $T(n) \in O(2^n)$ and $T(n) \in \Omega(2^n)$.

Assume that:
\begin{equation*}
\label{eq-expo}
T(k) \leq c2^k - ck, \forall k < n
\end{equation*}

We obtain that:
\begin{equation*}
\label{eq-expo2}
\begin{split}
T(n) & \leq 2c2^{n-1}+n-1 - 2c(n-1) \\
	&= c2^{n} + (n-1)(1 - 2c) \\
	& \leq c2^{n}
\end{split}
\end{equation*}

The last inequation folows if $(n-1)(1-2c) \leq 0$. For this choose $c=2$. Base case holds since $T(1) = 1 \leq 2 \cdot 2^{1} - 2\cdot 1=2$.

Assume that:
\begin{equation*}
\label{eq-expo3}
T(k) \geq c2^k, \forall k < n
\end{equation*}

We obtain that:
\begin{equation*}
\begin{split}
	T(n)
	 & \geq 2c2^{n-1}+n-1 \\
	 & \geq c2^{n} + n-1 \\
	 & \geq c2^{n}
\end{split}
\end{equation*}

This is true since $n \geq 1$. Now choose $c = 1$. Base case holds for $T(1) = 1 \geq 1 \cdot 2^{1} = 1$.

\noindent\textbf{212.} $T(1) = 3$. $T(n) = 4T(n/3) + 2n - 1$. $n = 3^p$, $p \geq 0$.

Using the tree method and summing on all levels we get that:
\begin{equation*}
	T(n) = \sum_{i = 0}^{\log_3 n - 1}((4/3)^i \cdot 2 n - 1) + 4^{\log_3 n}
\end{equation*}


Observe that the last level has $n^{\log_3 4}$ vertices. This will turn out a valid guess and we can prove using the substitution method that $T(n) \in O(n^{\log_3 4})$ as well as $T(n) \in \Omega(n^{\log_3 4})$ concluding with $T(n) \in \Theta(n^{\log_3 4})$.


\noindent\textbf{218.} $T(1) = 1$. $T(n) = 4T(n/3) + n^2$. $n = 3^p$, $p \geq 0$.

\begin{equation*}
\begin{split}
	\sum_{i = 0}^{\log_3 n}{(\frac{n}{3^i}^2 \cdot 4^i)} &= n^2 \sum_{i = 0}^{\log_3 n}{\frac{4^i}{3^{2i}}} \\
	&= n^2(1 - (4/9)^{\log_3 n}) / (1 - 4/9) \\
	&= n^2 (1 - (4/9)^{\log_3 n}) \cdot 5/9 \in \Theta(n^2)
\end{split}
\end{equation*}

\noindent\textbf{220.} $T(1) = 1$. $T(n) = T(n/4) + \sqrt{n} + 1$. $n = 4^k$, $k \geq 0$.

\begin{equation*}
\begin{split}
	\sum_{i=0}^{\log_4 n}(\sqrt n/2^i + 1) + \log_4 n= \sqrt n (2^n - 1) / 2^{n+1} + 2 \log_4 n
\end{split}
\end{equation*}
As $(2^n - 1) / 2^{n+1} \in (0, 1)$, $T(n) \in \Theta(\sqrt n)$
\end{document}