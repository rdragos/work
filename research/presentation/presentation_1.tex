%%%%%%%%%%%%%%%%%%%%%%%%%%%%%%%%%%%%%%%%%
% Beamer Presentation
% LaTeX Template
% Version 1.0 (10/11/12)
%
% This template has been downloaded from:
% http://www.LaTeXTemplates.com
%
% License:
% CC BY-NC-SA 3.0 (http://creativecommons.org/licenses/by-nc-sa/3.0/)
%
%%%%%%%%%%%%%%%%%%%%%%%%%%%%%%%%%%%%%%%%%

%----------------------------------------------------------------------------------------
%	PACKAGES AND THEMES
%----------------------------------------------------------------------------------------

\documentclass{beamer}

\mode<presentation> {

% The Beamer class comes with a number of default slide themes
% which change the colors and layouts of slides. Below this is a list
% of all the themes, uncomment each in turn to see what they look like.

%\usetheme{default}
%\usetheme{AnnArbor}
%\usetheme{Antibes}
%\usetheme{Bergen}
%\usetheme{Berkeley}
%\usetheme{Berlin}
%\usetheme{Boadilla}
%\usetheme{CambridgeUS}
%\usetheme{Copenhagen}
%\usetheme{Darmstadt}
%\usetheme{Dresden}
%\usetheme{Frankfurt}
%\usetheme{Goettingen}
%\usetheme{Hannover}
%\usetheme{Ilmenau}
%\usetheme{JuanLesPins}
%\usetheme{Luebeck}
\usetheme{Madrid}
%\usetheme{Malmoe}
%\usetheme{Marburg}
%\usetheme{Montpellier}
%\usetheme{PaloAlto}
%\usetheme{Pittsburgh}
%\usetheme{Rochester}
%\usetheme{Singapore}
%\usetheme{Szeged}
%\usetheme{Warsaw}

% As well as themes, the Beamer class has a number of color themes
% for any slide theme. Uncomment each of these in turn to see how it
% changes the colors of your current slide theme.

%\usecolortheme{albatross}
%\usecolortheme{beaver}
%\usecolortheme{beetle}
%\usecolortheme{crane}
%\usecolortheme{dolphin}
%\usecolortheme{dove}
%\usecolortheme{fly}
%\usecolortheme{lily}
%\usecolortheme{orchid}
%\usecolortheme{rose}
%\usecolortheme{seagull}
%\usecolortheme{seahorse}
%\usecolortheme{whale}
%\usecolortheme{wolverine}

%\setbeamertemplate{footline} % To remove the footer line in all slides uncomment this line
%\setbeamertemplate{footline}[page number] % To replace the footer line in all slides with a simple slide count uncomment this line

%\setbeamertemplate{navigation symbols}{} % To remove the navigation symbols from the bottom of all slides uncomment this line
}

\usepackage{graphicx} % Allows including images
\usepackage[absolute, overlay]{textpos}
\usepackage{booktabs} % Allows the use of \toprule, \midrule and \bottomrule in tables
\usepackage{tikz}
\usetikzlibrary{positioning,calc,backgrounds,shapes}

\usepackage[labelformat=empty]{caption}
\captionsetup{compatibility=false}

\tikzset{My Arrow Style/.style={single arrow, fill=red!30, anchor=base, align=center,text width=4cm}}
\newcommand{\arrowthis}[2][]{\tikz[baseline] \node [My Arrow Style,#1] {#2};}


\tikzset{My Speech Style/.style={ellipse callout, fill=red!50, anchor=base, align=center,text width=2.8cm}}
\newcommand{\speechthis}[2][]{
    \tikz[baseline]{\node[My Speech Style, #1]{#2};}
}%

\newcommand{\source}[1]{\begin{textblock*}{4cm}(8.7cm,8.2cm)
        \begin{beamercolorbox}[ht=0.5cm,right]{framesource}
        \usebeamerfont{framesource}\usebeamercolor[fg]{framesource} Source: {#1}
        \end{beamercolorbox}
    \end{textblock*}
}
%----------------------------------------------------------------------------------------
%	TITLE PAGE
%----------------------------------------------------------------------------------------

\title[University of Bucharest]{An Immediate Multi-Party Generalization of ID-NIKE from Constrained PRF}
% The short title appears at the bottom of every slide, the full title is only on the title page
%An Immediate Multi-Party Generalization of ID-NIKE from Constrained PRF
\author[R.F.Olimid, D.A.Rotaru]{Ruxandra F. Olimid and \textbf{Drago\c{s} Alin Rotaru}} % Your name
\institute[UniBuc] % Your institution as it will appear on the bottom of every slide, may be shorthand to save space
{
University of Bucharest\\ % Your institution for the title page
% \medskip
% \textit{ruxandra.olimid@fmi.unibuc.ro, r.dragos0@gmail.com} % Your email address
}
\date{September 16, 2014} % Date, can be changed to a custom date

\begin{document}

\begin{frame}
\titlepage % Print the title page as the first slide
\end{frame}


%----------------------------------------------------------------------------------------
%	PRESENTATION SLIDES
%----------------------------------------------------------------------------------------

%------------------------------------------------
\section{Overview} % Sections can be created in order to organize your presentation into discrete blocks, all sections and subsections are automatically printed in the table of contents as an overview of the talk
%------------------------------------------------

\begin{frame}
    \frametitle{Asymmetric Crypto Overview} % Table of contents slide, comment this block out to remove it

    \begin{textblock*}{5cm}(0cm, 3cm)
        \begin{figure}
            \includegraphics[width=5cm,height=3cm,keepaspectratio]{bob.jpg}
            \caption{Bob}
        \end{figure}
    \end{textblock*}

    \begin{textblock*}{2cm}(5.5cm, 1cm)
        \begin{figure}
            \includegraphics[width=1cm,height=1cm]{eve.png}
            \caption{Eve}
        \end{figure}
    \end{textblock*}

    \begin{textblock*}{5cm}(8cm, 3cm)
        \begin{figure}
            \includegraphics[width=5cm,height=3cm,keepaspectratio]{alice.jpeg}
            \caption{Alice}
        \end{figure}
    \end{textblock*}

    \begin{textblock*}{3cm}(4.5cm, 5cm)
        \only<2-5>{\arrowthis[]{$c \leftarrow E(pk, m)$}}
    \end{textblock*}

    \begin{textblock*}{3cm}(8cm, 2cm)
        \only<3-4>{\speechthis{I know the message!}}
    \end{textblock*}

    %decription for Alice
    \begin{textblock*}{4cm}(9.5cm, 8cm)
        \only<4>{$m \leftarrow D(sk, c)$}
    \end{textblock*}

    %speech for Eve
    \begin{textblock*}{0.5cm}(7cm, 1cm)
        \only<5>{\scalebox{10}{\textbf{?}}}
    \end{textblock*}

    \begin{textblock*}{3cm}(4.5cm, 5cm)
        \only<6>{\arrowthis[]{$c \leftarrow E(Alice@ex.com, m)$}}
    \end{textblock*}

\end{frame}

%------------------------------------------------
\section{Constrained PRF}
%------------------------------------------------

\def\firstcircle{(0,0) circle (1.5cm)}
\def\secondcircle{(0,-1) circle (0.5cm)}

\begin{frame}
\frametitle{Constrained PRF}

    \begin{definition} {
            A PRF is a function $F : \mathcal{K} \times \mathcal{X} \rightarrow \mathcal{Y}$ such that there is
            a polynomial algorithm to evaluate $F(k, \cdot)$, $k \in \mathcal{K}$
        }
    \end{definition}

    \begin{itemize}
        \item
            A constrained PRF (cPRF) is similar to a PRF, with an additional set of constrained keys\
        $\mathcal{K}_c$ such that a key $k_s \in \mathcal{K}_c$ enables the evaluation of\
            $F$ only in a certain subset $S$ $\in$ $\mathcal{X}$.
    \end{itemize}
    \begin{figure}
        \begin{tikzpicture}
        \begin{scope}[shift={(3cm,-5cm)}, fill opacity=0.5, align=center]
            \fill[blue]{(0, 0) circle(1.5cm)};
            \draw \firstcircle node[left]{$F(K, \cdot)$};
            \draw \secondcircle node[]{$S$};
        \end{scope}
        \end{tikzpicture}

    \end{figure}

\end{frame}


%------------------------------------------------
\subsection{Left / Right cPRF}
\begin{frame}
    \frametitle{Left/Right cPRF, Bit-Fixing cPRF}

       \begin{definition}
        Let $F : \mathcal{K} \times \mathcal{X}^2 \rightarrow \mathcal{Y}$ be a PRF. Then, $\forall w \in \mathcal{X}$,\
        a left/right cPRF supports two constrained keys $k_w^L$ and $k_w^R$\
        that enable the evaluation of $F$ at all points\
        $(w,x) \in \mathcal{X}^2$, respectively $(x,w) \in \mathcal{X}^2$.
        \end{definition}

	\pause
        \begin{definition}
        Let $F : \mathcal{K} \times \{0,1\}^N \rightarrow \mathcal{Y}$ be a PRF.\
        Then, $\forall v \in \{0,1,?\}^N$, a bit-fixing cPRF  supports a constrained\
        key $k_v$ that enables the evaluation of $F$ at all points $x \in \{0,1\}^N$ that satisfy the pattern $v$.
        \end{definition}

	\pause
        \begin{exampleblock}{Example}
            When $v:={0?1}$, $k_{0?1}$ enables the evalation of $F(k_{0?1}, 011)$ and $F(k_{0?1}, 001)$
        \end{exampleblock}

\end{frame}

%------------------------------------------------


%------------------------------------------------
\subsection{Boneh Waters ID-NIKE [BW'13]}

\begin{frame}
    \frametitle{Boneh-Waters ID-NIKE [BW'13]}

    \only<1->{
        \begin{textblock*}{5cm}(0cm, 3cm)
        \begin{figure}
            \includegraphics[width=5cm,height=3cm,keepaspectratio]{bob.jpg}
            \caption{Bob}
        \end{figure}
        \end{textblock*}

        \begin{textblock*}{5cm}(8cm, 3cm)
            \begin{figure}
                \includegraphics[width=5cm,height=3cm,keepaspectratio]{alice.jpeg}
                \caption{Alice}
            \end{figure}
        \end{textblock*}
    }
    \only<2->{
        \begin{textblock*}{2cm}(3cm,4.5cm)
            $F(k_{Bob}, (Bob, \cdot))$
            $F(k_{Bob}, (\cdot, Bob))$
        \end{textblock*}
    }

    \only<3->{
        \begin{textblock*}{2cm}(6.5cm,4.5cm)
            $F(k_{Alice}, (Alice, \cdot))$
            $F(k_{Alice}, (\cdot, Alice))$
        \end{textblock*}
    }
    \only<4->{
        \begin{textblock*}{6cm}(4cm, 2cm)
            We have a common secret key: $F(msk, (Alice, Bob))$
        \end{textblock*}
    }


\end{frame}


%-----------------
\begin{frame}
  \frametitle{Boneh-Waters ID-NIKE [BW'13]}

\begin{itemize}
	\item $\mathrm{Setup}(\lambda)$:
		\begin{itemize}
			\item  let $F : \mathcal{K} \times \mathcal{X}^2 \rightarrow \mathcal{Y}$ be $PRF^{L/R}$, $msk \leftarrow^R \mathcal{K}$
			\item outputs $msk$
		\end{itemize}
	\item $\mathrm{Extract}(msk,id_i)$:
		\begin{itemize}
			\item  computes $\mathrm{F.constrain}(msk, \{(id_i, \cdot)\})$ to obtain $k^{L}_{id_i}$ and  $\mathrm{F.constrain}(msk, \{(\cdot, id_i)\})$  to obtain $k^{R}_{id_i}$
			\item outputs $sk_{id_i} = (k^{L}_{id_i},k^{R}_{id_i})$
		\end{itemize}
	\item $\mathrm{KeyGen}(sk_{id_i},id_j)$ outputs:

\[ k_{id_i,id_j} = \left\{ \begin{array}{lll}
F(msk, (id_i, id_j)) & \mbox{if} & id_i < id_j \\
F(msk, (id_j, id_i)) & \mbox{if} & id_i > id_j
\end{array}\right.
\]

\end{itemize}

\end{frame}


%------------------------------------------------
\section{Boneh-Waters ID NIKE construction & Our generalization}

%------------------------------------------------
\begin{frame}
    \frametitle{Multi-Party ID-NIKE from cPRF}
    \only<1->{

        \begin{textblock*}{4cm}(0cm, 3cm)
        \begin{figure}
            \includegraphics[width=5cm,height=3cm,keepaspectratio]{bob.jpg}
            \caption{Bob}
        \end{figure}
        \end{textblock*}

        \begin{textblock*}{2cm}(1cm,2cm)
            $F(k_{Bob}, (Bob, \cdot, \cdot))$
            $F(k_{Bob}, (\cdot, Bob, \cdot))$
            $F(k_{Bob}, (\cdot, \cdot, Bob))$
        \end{textblock*}

        \begin{textblock*}{2cm}(9cm,2cm)
            $F(k_{Alice}, (Alice, \cdot, \cdot))$
            $F(k_{Alice}, (\cdot, Alice, \cdot))$
            $F(k_{Alice}, (\cdot, \cdot, Alice))$
        \end{textblock*}

        \begin{textblock*}{4cm}(8cm, 3cm)
            \begin{figure}
                \includegraphics[width=5cm,height=3cm,keepaspectratio]{alice.jpeg}
                \caption{Alice}
            \end{figure}
        \end{textblock*}
    }
    \only<2->{
        \begin{textblock*}{4cm}(4.5cm, 3cm)
        \begin{figure}
            \includegraphics[width=5cm,height=3cm,keepaspectratio]{stickman-in-love-hi.png}
            \caption{John}
        \end{figure}
        \end{textblock*}
    }
    \only<3->{
        \begin{textblock*}{2cm}(5cm, 2cm)
            $F(k_{John}, (John, \cdot, \cdot))$
            $F(k_{John}, (\cdot, John, \cdot))$
            $F(k_{John}, (\cdot, \cdot, John))$
        \end{textblock*}
    }
    \only<4->{
        \begin{textblock*}{3cm}(5cm, 8cm)
            Secret Shared Key:
            $F(msk, (Alice, Bob, John))$
        \end{textblock*}
    }


\end{frame}

\begin{frame}
    \frametitle{Multi-Party ID-NIKE from cPRF}

\begin{itemize}
	\item $\mathrm{Setup}(\lambda)$:
		\begin{itemize}
			\item  let $F : \mathcal{K} \times \mathcal{X}^N \rightarrow \mathcal{Y}$ be $PRF^{bf}$, $msk \leftarrow^R \mathcal{K}$
			\item outputs $msk$
		\end{itemize}
	\item $\mathrm{Extract}(msk,id_i)$:
		\begin{itemize}
			\item  computes $\mathrm{F.constrain}(msk, \{(id_i, \cdot, \dots, \cdot)\})$ to obtain $k^{1}_{id_i}$, $\mathrm{F.constrain}(msk, \{(\cdot, id_i, \cdot, \dots, \cdot)\})$  to obtain $k^{2}_{id_i}$, $\cdots$,  $\mathrm{F.constrain}(msk, \{(\cdot, \dots, \cdot, id_i)\})$ to obtain $k^{N}_{id_i}$
			\item outputs $sk_{id_i} = (k^{1}_{id_i}, \dots, k^{N}_{id_i})$
		\end{itemize}
	\item $\mathrm{KeyGen}(sk_{id_i},\{id_1, \dots, id_N\})$ outputs:


\[ k_{id_1, \dots, id_N} = F(msk, (id_{\pi(1)}, id_{\pi(2)}, \dots, id_{\pi(N)})) \]

where $id_{\pi(1)} < id_{\pi(2)} < \dots < id_{\pi(N)}$ (in lexicographic order)

\end{itemize}

\end{frame}
%------------------------------------------------

\begin{frame}
\Huge{\centerline{Thank you!}}
\end{frame}

\begin{frame}
    \frametitle{Questions}
    \url{http://eprint.iacr.org/2013/352.pdf} page[7]
\end{frame}

\section{cPRF construction}
%----------------------------------------------------------------------------------------

\end{document}
