%%%%%%%%%%%%%%%%%%%%%%%%%%%%%%%%%%%%%%%%%
% Beamer Presentation
% LaTeX Template
% Version 1.0 (10/11/12)
%
% This template has been downloaded from:
% http://www.LaTeXTemplates.com
%
% License:
% CC BY-NC-SA 3.0 (http://creativecommons.org/licenses/by-nc-sa/3.0/)
%
%%%%%%%%%%%%%%%%%%%%%%%%%%%%%%%%%%%%%%%%%

%----------------------------------------------------------------------------------------
%	PACKAGES AND THEMES
%----------------------------------------------------------------------------------------

\documentclass{beamer}

\mode<presentation> {

% The Beamer class comes with a number of default slide themes
% which change the colors and layouts of slides. Below this is a list
% of all the themes, uncomment each in turn to see what they look like.

%\usetheme{default}
%\usetheme{AnnArbor}
%\usetheme{Antibes}
%\usetheme{Bergen}
%\usetheme{Berkeley}
%\usetheme{Berlin}
%\usetheme{Boadilla}
%\usetheme{CambridgeUS}
%\usetheme{Copenhagen}
%\usetheme{Darmstadt}
%\usetheme{Dresden}
%\usetheme{Frankfurt}
%\usetheme{Goettingen}
%\usetheme{Hannover}
%\usetheme{Ilmenau}
%\usetheme{JuanLesPins}
%\usetheme{Luebeck}
\usetheme{Madrid}
%\usetheme{Malmoe}
%\usetheme{Marburg}
%\usetheme{Montpellier}
%\usetheme{PaloAlto}
%\usetheme{Pittsburgh}
%\usetheme{Rochester}
%\usetheme{Singapore}
%\usetheme{Szeged}
%\usetheme{Warsaw}

% As well as themes, the Beamer class has a number of color themes
% for any slide theme. Uncomment each of these in turn to see how it
% changes the colors of your current slide theme.

%\usecolortheme{albatross}
%\usecolortheme{beaver}
%\usecolortheme{beetle}
%\usecolortheme{crane}
%\usecolortheme{dolphin}
%\usecolortheme{dove}
%\usecolortheme{fly}
%\usecolortheme{lily}
%\usecolortheme{orchid}
%\usecolortheme{rose}
%\usecolortheme{seagull}
%\usecolortheme{seahorse}
%\usecolortheme{whale}
%\usecolortheme{wolverine}

%\setbeamertemplate{footline} % To remove the footer line in all slides uncomment this line
%\setbeamertemplate{footline}[page number] % To replace the footer line in all slides with a simple slide count uncomment this line

%\setbeamertemplate{navigation symbols}{} % To remove the navigation symbols from the bottom of all slides uncomment this line
}

\usepackage{graphicx} % Allows including images
\usepackage[absolute, overlay]{textpos}
\usepackage{booktabs} % Allows the use of \toprule, \midrule and \bottomrule in tables
\usepackage{tikz}
\usetikzlibrary{positioning,calc,backgrounds,shapes}

\usepackage[labelformat=empty]{caption}
\captionsetup{compatibility=false}

\tikzset{My Arrow Style/.style={single arrow, fill=red!30, anchor=base, align=center,text width=4cm}}
\newcommand{\arrowthis}[2][]{\tikz[baseline] \node [My Arrow Style,#1] {#2};}


\tikzset{My Speech Style/.style={ellipse callout, fill=red!50, anchor=base, align=center,text width=2.8cm}}
\newcommand{\speechthis}[2][]{
    \tikz[baseline]{\node[My Speech Style, #1]{#2};}
}%

\newcommand{\source}[1]{\begin{textblock*}{4cm}(8.7cm,8.2cm)
        \begin{beamercolorbox}[ht=0.5cm,right]{framesource}
        \usebeamerfont{framesource}\usebeamercolor[fg]{framesource} Source: {#1}
        \end{beamercolorbox}
    \end{textblock*}
}
%----------------------------------------------------------------------------------------
%	TITLE PAGE
%----------------------------------------------------------------------------------------

\title[University of Bucharest]{Graphene Quantum Dots Interfaced with Single Bacterial Spore for Bio-Electromechanical Devices: A Graphene Cytobot}
% The short title appears at the bottom of every slide, the full title is only on the title page
%An Immediate Multi-Party Generalization of ID-NIKE from Constrained PRF
\author[Drago\c{s} Alin Rotaru]{Drago\c{s} Alin Rotaru} % Your name
\institute[UniBuc] % Your institution as it will appear on the bottom of every slide, may be shorthand to save space
{
University of Bucharest\\ % Your institution for the title page
% \medskip
}
\date{26 March, 2015} % Date, can be changed to a custom date

\begin{document}

\begin{frame}
\titlepage % Print the title page as the first slide
\end{frame}


%----------------------------------------------------------------------------------------
%	PRESENTATION SLIDES
%----------------------------------------------------------------------------------------

%------------------------------------------------
\section{Basic Idea} % Sections can be created in order to organize your presentation into discrete blocks, all sections and subsections are automatically printed in the table of contents as an overview of the talk
%------------------------------------------------

\begin{frame}
    \frametitle{Measure the humidity of the environment} % Table of contents slide, comment this block out to remove it
    \begin{itemize}
        \item Take some GQD (Graphene Quantum Dots) and place them on a bacteria spore
        \pause
        \item Attach electrods to the GQD to measure their conductivity
        \pause
        \item Humidity drops and the spore shrinks, mainly because the water is pushed out from the spore
        \pause
        \item Because the spore shrinks, the GQD's increase their conductivity according to the electrodes
    \end{itemize}
\end{frame}

\begin{frame}
    \frametitle{Examples}
    \begin{textblock*}{5cm}(2cm, 3cm)
        \begin{figure}
            \includegraphics[width=9cm,height=9cm,keepaspectratio]{example.jpg}
            \caption{Spore with GQD}
        \end{figure}
    \end{textblock*}
\end{frame}

\begin{frame}
    \frametitle{About Graphene}
    \begin{textblock*}{8cm}(2cm, 2cm)
        \begin{figure}
            \includegraphics[width=6cm,height=8cm,keepaspectratio]{graphene.jpg}
            \caption{Graphene. Honeycomb lattice at atomic scale}
        \end{figure}
    \end{textblock*}
\end{frame}

\begin{frame}
    \frametitle{About Quantum Dots}
    \begin{itemize}
        \item Small nanocrystal made of semiconductor materials
        \pause
        \item Exhibit quantum properties
    \end{itemize}
\end{frame}
%----------------------------------------------------------------------------------------

\end{document}
